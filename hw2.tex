\documentclass[12pt,a4paper]{article}
\usepackage{amsmath}
\usepackage{amsfonts}
\usepackage{amssymb}
\usepackage{apacite}
\usepackage{graphicx}
\usepackage[top=2.5cm, bottom=2.5cm, left=2.5cm, right=2.5cm]{geometry}
\usepackage[toc,page]{appendix}
\usepackage{hyperref}
\usepackage{fancyref}
\usepackage[round]{natbib}
\usepackage{fancyhdr}
\pagestyle{fancy}

\lfoot{Pragmatics HW \#2}
\rfoot{Word count: 495}

\begin{document}

\title{Pragmatics Homework \#2: Indirect speech acts and implicature}
\author{Exam Number B018520}

\maketitle

\begin{enumerate}


\item The temporal use of ``and" as demonstrated in (i) appears to be a case of generalized conversational implicature. This judgment stems from my observations that the temporal reading of ``and" does not appear to be dependent on a specific given context (and ergo not an instance of particularized conversational implicature), nor is this reading of ``and" difficult to cancel (and ergo not an instance of conventional implicature). 

As evidence for the former, consider the following sentence:
\begin{quote}
Add the chopped vegetables and bring the broth to a boil.
\end{quote}
Without the need for a specific context (e.g. ``Perform these steps in the exact order I say"), in my evaluation the natural reading of this sentence carries the implicature that the chopped vegetables should first be added (to, presumably, a soup of some kind), after which point the broth should be brought to a boil. Conversely, someone performing these actions the other way around (first bringing the broth to a boil, then adding vegetables) would seem to have severely misapprehended the instructions. 

It might be argued that there is a natural temporal relationship between adding vegetables to a dish and bringing the dish to a boil, in that in many common recipes the dish is brought to a boil after all or nearly all ingredients have been added. To further support the claim that the temporal use of ``and" is not particularized, we can consider the following two discourses featuring events with no immediate causal relation:
\begin{quote}
Alice: I talked to Mallory briefly, and went to see the movie.\\
Bob: \# Did you tell her what you thought of it?
\end{quote}
\begin{quote}
Alice: I went to see the movie, and talked to Mallory briefly.\\
Bob: \# I hope she didn't spoil anything for you!
\end{quote}
In my evaluation, in both these contexts Bob seems to have misinterpreted the implied order of events, which has been established without the need for a particular context outside Alice's utterances. By contrast, we can add a context which dispels the temporal reading of ``and", a hallmark of generalized conversational implicatures:
\begin{quote}
Alice: I bumped into Mallory the other day! I went to see the movie, and talked to her briefly.\\
Bob: Oh, was she going to see it as well?\\
Alice: Yes, we ended up sitting together.
\end{quote}

As mentioned previously, the temporal use of ``and" can also be canceled with ease, providing evidence against its being a conventional implicature. Consider the following example:
\begin{quote}
He'll probably move away, and we'll never see him again. Although, we might never see him again even before he moves away!
\end{quote}
In this case, the implicit temporal relationship between the person moving away and never being seen is negated without trouble. This can likewise be demonstrated in the following example:
\begin{quote}
Add the chopped vegetables and bring the broth to a boil, although not necessarily in this order.
\end{quote}

\end{enumerate}


\end{document}
