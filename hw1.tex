\documentclass[12pt,a4paper]{article}
\usepackage{amsmath}
\usepackage{apacite}
\usepackage{graphicx}
\usepackage[top=2.5cm, bottom=2.5cm, left=2.5cm, right=2.5cm]{geometry}
\usepackage[toc,page]{appendix}
\usepackage{hyperref}
\usepackage{fancyref}
\usepackage[round]{natbib}

\begin{document}

\title{Pragmatics Homework \#1: Presuppositions}
\author{Exam Number B018520}

\maketitle

\part{}
\begin{enumerate}

\item \textbf{Propositions b. and c. are presupposed}, while \textbf{proposition d. is entailed}, assuming 
\[
\forall x.\,\diamondsuit scared(x) \rightarrow animate(x)
\]
My reasoning for \textbf{b.} is as follows: consider the negation test, i.e. 
\begin{equation}
\text{That John was assaulted did not scare Mary.}
\end{equation}
as applied to an inanimate object, e.g.
\begin{equation}
\#\;\text{That John was assaulted did not scare the table.}\footnote{The symbol $\#$ throughout this work is used to indicate my own evaluation.}
\end{equation}
Sentence 2 in my evaluation implies a table capable of being scared. To support this, a context can be constructed which in my evaluation triggers accommodation of the table being animate, e.g.
\begin{quote}
Bursts of light erupted from Mickey's wand as he made the broomsticks dance; this did not scare the table.
\end{quote}
My reasoning for \textbf{c.} stems primarily from the negation test as performed in (1): John's assault survives negation. The contrapositive test can be used to show \textbf{a.}'s entailment relationship with \textbf{d.}. Given
\begin{quote}
That John was assaulted did not cause fear in Mary.
\end{quote}
Mary was also not scared by John's assault, assuming causing fear in and being scared are roughly synonymous.

\begin{quote}
That's false; Carmen still works at the University of Edinburgh.
\end{quote}
with
\begin{quote}
\# That's false; Carmen never worked at the University of Edinburgh.
\end{quote}
Contraposition holds for \textbf{c.} and \textbf{a.}, demonstrating an entailment relationship:
\begin{quote}
It's not the case that Carmen is not working at the University of Edinburgh.
\therefore
It's not the case that Carmen is no longer working at the University of Edinburgh.
\end{quote}


\item \textbf{Proposition b. is presupposed; proposition c. is entailed.}
\end{enumerate}


\part{}
\begin{enumerate}
\setcounter{enumi}{3}
\item \textbf{Proposition a. entails b.}, and vice-versa. Simply stated,
\[
\neg \forall x.\, one(x) \rightarrow try(x, kill(Templeton, x))
\]
is equivalent to
\[
\exists x.\, one(x) \wedge \neg try(x, kill(Templeton, x))
\]
following from the well-established equivalences
\[
\neg \forall x.\, P(x) \equiv \exists x.\, \neg P(x)
\]
and
\[
\neg (p \rightarrow q) \equiv p \wedge \neg q
\]

\item \textbf{Proposition b. entails a.}

\item \textbf{Proposition a. presupposes b.}
\end{enumerate}


\part{}
\begin{enumerate}
\setcounter{enumi}{7}
\begin{enumerate}

\item 

\end{enumerate}
\end{enumerate}


\end{document}
