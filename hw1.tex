\documentclass[12pt,a4paper]{article}
\usepackage{amsmath}
\usepackage{amsfonts}
\usepackage{amssymb}
\usepackage{apacite}
\usepackage{graphicx}
\usepackage[top=2.5cm, bottom=2.5cm, left=2.5cm, right=2.5cm]{geometry}
\usepackage[toc,page]{appendix}
\usepackage{hyperref}
\usepackage{fancyref}
\usepackage[round]{natbib}

\begin{document}

\title{Pragmatics Homework \#1: Presuppositions}
\author{Exam Number B018520}

\maketitle

\part{}
\begin{enumerate}

\item \textbf{Propositions b. and c. are presupposed}, while \textbf{proposition d. is entailed}, assuming 
\[
\forall x.\,\diamondsuit scared(x) \rightarrow animate(x)
\]
My reasoning for \textbf{b.} is as follows: consider the negation test, i.e. 
\begin{equation}
\text{That John was assaulted did not scare Mary.}
\end{equation}
as applied to an inanimate object, e.g.
\begin{equation}
\#\;\text{That John was assaulted did not scare the table.}\footnote{The symbol $\#$ throughout this work is used to indicate my own evaluation.}
\end{equation}
Sentence 2 implies a table capable of being scared. To support this, a context can be constructed which in my evaluation triggers accommodation of the table being animate, e.g.
\begin{quote}
Bursts of light erupted from Mickey's wand as he made the broomsticks dance; this did not scare the table.
\end{quote}
My reasoning for \textbf{c.} stems primarily from the negation test as performed in (1): John's assault survives negation. The contrapositive test can be used to show \textbf{a.}'s entailment relationship with \textbf{d.}. Given
\begin{quote}
That John was assaulted did not cause fear in Mary.
\end{quote}
Mary was also not scared by John's assault, assuming causing fear in and being scared are roughly synonymous.

\item \textbf{Proposition b. is presupposed; proposition c. is entailed.} The presupposition relationship of \textbf{a.} to \textbf{b.} can be demonstrated using a denial test: contrast
\begin{quote}
That's false; Carmen still works at the University of Edinburgh.
\end{quote}
with
\begin{quote}
\# That's false; Carmen never worked at the University of Edinburgh.
\end{quote}
Contraposition holds for \textbf{c.} and \textbf{a.}, demonstrating an entailment relationship:
\begin{quote}
It's not the case that Carmen is not working at the University of Edinburgh. \\
$\therefore$ \\
It's not the case that Carmen is no longer working at the University of Edinburgh.
\end{quote}


\item \textbf{Proposition b. is presupposed; proposition c. is entailed.} The presupposition relationship with \textbf{b.} can be demonstrated via the negation test: in
\begin{quote}
John managed to get the job.
\end{quote}
John's finding it difficult to get the job survives. It should be noted, however, that this may be partially contingent on how strictly the use of ``managed" is constrained by the difficulty of the task; this is to say, in my evaluation, this could plausibly be considered a case of implicature, e.g.
\begin{quote}
Did John manage to get the job?\\
Yes, he managed to get it, in fact, he found it quite easy.
\end{quote}
Contraposition can again be used to demonstrate that \textbf{a.} entails \textbf{c.}, e.g.
\begin{quote}
It's not the case that John didn't get the job.\\
$\therefore$\\
It's not the case that John didn't manage to get the job.
\end{quote}

\end{enumerate}


\part{}
\begin{enumerate}
\setcounter{enumi}{3}
\item \textbf{Proposition a. entails b.}, and vice-versa. Simply stated,
\[
\neg \forall x.\, one(x) \rightarrow try(x, kill(Templeton, x))
\]
is equivalent to
\[
\exists x.\, one(x) \wedge \neg try(x, kill(Templeton, x))
\]
following from the well-established equivalences
\[
\neg \forall x.\, P(x) \equiv \exists x.\, \neg P(x)
\]
and
\[
\neg (p \rightarrow q) \equiv p \wedge \neg q
\]

\item \textbf{Proposition b. entails a.}, as demonstrated by contraposition:
\begin{quote}
It's not the case that someone cheated on the exam.\\
$\therefore$\\
It's not the case that John cheated on the exam.
\end{quote}

\item \textbf{Proposition a. presupposes b.}. This can be clearly demonstrated by using the negation test: in
\begin{quote}
It's not the case that if John realizes that Mary is in New York, he will get angry.
\end{quote}
, proposition \textbf{b.} survives. Alternatively, this can also be demonstrated using the denial test, by contrasting the following responses to \textbf{a.}:
\begin{quote}
No, John won't get angry.\\
\# No, Mary isn't in New York.
\end{quote}
\end{enumerate}


\part{}
\begin{enumerate}
\setcounter{enumi}{7}
\item 
\begin{enumerate}

\item 
\begin{enumerate}
\item Baldness exists.
\item Heredity exists.
\item France exists.
\item The king of France exists.
\end{enumerate}
All of the above are projected presuppositions, as demonstrated by the following denial examples:
\begin{quote}
\# No, there isn't such a thing as baldness.\\
\# No, there isn't such a thing as heredity.\\
\# No, France isn't a place.\\
\# No, there is no king of France.
\end{quote}

\item
The presuppositions from the former subclause are as follows:
\begin{enumerate}
\item France exists.
\end{enumerate}
The presuppositions from the latter are:
\begin{enumerate}
\item France exists.
\item The king of France exists.
\item Baldness exists.
\end{enumerate}
Of the presuppositions in both the former and latter subclauses, only ii. from the latter is not projected - a reasonable response might take a form along the lines of
\begin{quote}
I don't think there is a king of France.
\end{quote}
without, in my evaluation, derailing the discourse.

\item
\begin{enumerate}
\item ``I" (the speaker) exist.\footnote{Admittedly, a bit of a given.}
\item ``You" (the listener) exist.
\item Pragmatics exists.
\item ``You" are going to Pragmatics.
\end{enumerate}
Of these, presuppositions iii. and iv., which stem from the subclause ``stop going to Pragmatics.", are projected into the larger utterance, as evidenced through the denial test:
\begin{quote}
\# No, Pragmatics isn't being offered.\\
\# No, I don't go to Pragmatics.
\end{quote}

\item
From the former subclause:
\begin{enumerate}
\item John exists.
\item Breakfast exists.
\end{enumerate}
From the latter subclause:
\begin{enumerate}
\item ``He" (John) exists.
\item Breakfast exists, as linked to from the former subclause.
\item Donuts exist.
\item John, at least one point, ate donuts for breakfast.
\end{enumerate}
Only iv. from the latter subclause is not projected, and can be readily accessed in a response by the listener:
\begin{quote}
I don't think John did ever eat donuts, no.
\end{quote} 

\item
From the former subclause:
\begin{enumerate}
\item John exists.
\item John exercises.\footnote{This, arguably, could instead be a case of implicature: ``Yes, you could say John is exercising more - he didn't exercise in the first place".}
\end{enumerate}
From the latter:
\begin{enumerate}
\item ``He" (John) exists.
\item Donuts exist.
\item Breakfast exists.
\item John, at least at one, point ate donuts for breakfast.
\end{enumerate}
In this instance, all presuppositions are projected.

\end{enumerate}
Of the above, only \textbf{b.} and \textbf{d.} are examples where a presupposition from a subclause is prevented from being projected to the utterance as a whole. In both cases, a logical structure similar to 
\[
\neg P \oplus (P \wedge Q)
\]
is employed. This is to say, \textbf{d.} might be plausibly formulated in first-order logic as
\[
\neg eaten(John, donuts) \oplus (eaten(John, donuts) \wedge \neg eats(John, donuts))
\],
and \textbf{b.}'s logical form might resemble
\[
\neg (\exists x. king(x, France)) \oplus (\exists x. king(x, France) \wedge bald(x))
\]
In both instances, an initial ``out" is supplied in the initial subclause, which allows the listener to select the negation of a presupposition from the latter subclause.

\end{enumerate}


\end{document}
