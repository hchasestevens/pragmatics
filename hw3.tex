\documentclass[12pt,a4paper]{article}
\usepackage{amsmath}
\usepackage{amsfonts}
\usepackage{amssymb}
\usepackage{apacite}
\usepackage{graphicx}
\usepackage[top=2.5cm, bottom=2.5cm, left=2.5cm, right=2.5cm]{geometry}
\usepackage[toc,page]{appendix}
\usepackage{hyperref}
\usepackage{fancyref}
\usepackage[round]{natbib}
\usepackage{fancyhdr}
\pagestyle{fancy}

\fancyhead{}
\lfoot{Pragmatics HW \#3}
\rfoot{Word count: 400}

\begin{document}

\title{Pragmatics Homework \#3: Information Structure: Clefts}
\author{Exam Number B018520}

\maketitle

\section{Examples}

\begin{enumerate}

\item ``[...] It was they who created a different symbol for every number from one to nine. [...]"

\item ``It was in 1971 that someone looked out his window and saw five bright objects. [...]"

\item ``[...] It was actually my sister who said we should go [...]"

\item ``[... It] was at a radiology conference in 1974 that a physician named Lee Rogers, who had inadvertently swallowed a can tab during a basketball game, broached the idea of the can tab hazard to the medical community."

\item ``[... It] was my husband's father who did the cooking."

\item ``[...] It was a potential buyer who noticed it."

\end{enumerate}

\section{Analysis of examples}
\subsection{Analysis}
\begin{tabular}{ l | l | l | l }
Example & It was $X$ & that / who $Y$ & Proposed group \\
\hline
1 & Old        & New        & 1 \\
2 & New        & New        & 2 \\
3 & Inferrable & Inferrable & 1 \\
4 & New        & New        & 2 \\
5 & Inferrable & Inferrable & 1 \\
6 & New        & Old        & 2 \\  
\end{tabular}
\subsection{Reasoning}
\begin{enumerate}

\item The pronoun ``they" necessarily refers to a discourse-old referent, in this case, the Indians. The use of the indefinite article ``a" in ``a different symbol" indicates that ``symbol" is discourse-new, as opposed to being inferrable from the discussion of numbers.

\item As this utterance has no previous context, both referents are necessarily new. 

\item The use of the personal pronoun generally indicates an old or inferrable referent; in this case, as the speaker's sister was not previously mentioned but can be easily accommodated\footnote{In my personal evaluation.}, she can be considered inferrable. As ``we" refers to two old referents (the speaker and the speaker's sister), this can either be considered old or inferrable (since the sister/speaker grouping has not explicitly been established beforehand). 

\item Here again the use of the indefinite article ``a" in ``a radiology conference" and ``a physician" establishes these as new referents.

\item It is inferrable that the speaker's discourse-old husband has a father; likewise, it is inferrable that a feast requires cooking beforehand.

\item The ``potential buyer" has not been mentioned previously and is introduced with an indefinite article, while the pronoun ``it" refers to the previously-mentioned disparity.

\end{enumerate}


\section{Questions}
\begin{enumerate}

\item In Group 1, the governing rule appears to be that \[
age(X) \geq age(Y)
\], whereas in Group 2, \[
age(X) = NEW
\].

\item The need to put emphasis on the subject of an utterance appears to license the use of old or inferrable referents in the $X$ position, as seen in Group 1. In examples 3 and 5, ``actually" and ``however" (respectively) are used by the speaker to draw the reader's attention to the subject in the $X$ position and to give some indication that the utterance is conveying new and surprising information. While this is not explicated as directly in example 1, the sentence directly following the example contrasts the Indian invention of our numeric system with their description as ``Arabic" numerals; if one assumes that the common ground before example 1 contained the fact that the symbols for one through nine are known as ``Arabic numerals", example 1 could plausibly be rephrased as ``It was they [, not the Arabs,] who created a different symbol [...]".

\end{enumerate}

\end{document}
